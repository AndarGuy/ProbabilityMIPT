\section{Базовые определения}

\begin{definition}
	Система $\mathcal{F}$ подмножеств $\Omega$ называется алгеброй, если
	\begin{enumerate}
		\item $\Omega \in \mathcal{F}$
		\item $A \in \mathcal{F}$, то $\overline{A} := (\Omega \setminus A) \in \mathcal{F}$
		\item $A,\, B \in \mathcal{F}$, то $A \cap B \in \mathcal{F}$
	\end{enumerate}
\end{definition}

\begin{definition}
	Система $\mathcal{F}$ подмножеств $\Omega$ называется $\sigma$-алгеброй, если
	\begin{enumerate}
		\item $\mathcal{F}$ -- алгебра
		\item $\forall \{A_n,\, n \in \mathbb{N}\},\, A_n \in \mathcal{F} \Rightarrow \cup_{n = 1}^\infty A_n \in \mathcal{F}$
	\end{enumerate}
\end{definition}

\begin{definition}
	$P$ называется вероятностной мерой на $(\Omega,\, \mathcal{F})$, если $P:\: \mathcal{F} \to [0,\,1]$, удовлетворяющая свойствам:
	\begin{enumerate}
		\item $P(\Omega) = 1$
		\item Если $\{A_n,\, n \in \mathbb{N}\}$, то
		      \[P\left(\bigsqcup_{n = 1}^\infty A_n\right) = \sum_{n = 1}^\infty P(A_n)\]
	\end{enumerate}
\end{definition}

\begin{definition}
	Вероятностное пространство -- это тройка $(\Omega,\, \mathcal{F},\, P)$, где
	\begin{itemize}
		\item $\Omega$ -- множество элементарных исходов
		\item $\mathcal{F}$ -- $\sigma$-алгебра подмножеств $\Omega$, элементы $\mathcal{F}$ называются событиями
		\item $P$ -- вероятностная мера на измеримом пространстве $(\Omega,\, \mathcal{F})$
	\end{itemize}
\end{definition}

\begin{definition}
	Система $\mathcal{M}$ подмножеств в $\Omega$ называется $\pi$-системой, если из того, что $A,\, B \in \mathcal{M}$ следует, что $A \cap B \in \mathcal{M}$
\end{definition}

\begin{definition}
	Система $\mathcal{L}$ подмножеств в $\Omega$ называется $\lambda$-системой, если
	\begin{enumerate}
		\item $\Omega \in \mathcal{L}$
		\item $(A,\, B \in \mathcal{L};\; A \subset B) \Rightarrow B \setminus A \in \mathcal{L}$
		\item $(A_n \uparrow A;\; \forall n \: A_n \in \mathcal{L}) \Rightarrow A \in \mathcal{L}$
	\end{enumerate}
\end{definition}

\begin{theorem} \label{FIRST_SYSTEM_TH}
	Первая теорема о $\pi$-$\lambda$-системах

	Система $\mathcal{F}$ подмножеств $\Omega$ является $\sigma$-алгеброй $\Leftrightarrow$ она является $\pi$-системой и $\lambda$-системой.
\end{theorem}

\begin{proof}
	$\Rightarrow$ Свойство $\pi$-системы и свойство 1) $\lambda$-системы выполняются автоматически.

	Рассмотрим $\forall A_n \uparrow A;\; \forall n \: A_n \in \mathcal{L}$. Тогда $\cup_{n = 1}^{\inf} = A \in \mathcal{L}$. Следовательно, выполнено свойство 3) $\lambda$-системы.

	$\forall A,\, B \in \mathcal{L};\; A \subset B : B \setminus A = B \cap \overline{A}$. Но $\overline{A} \in \mathcal{L}$, следовательно $B \cap \overline{A} \in \mathcal{L}$.
	То есть выполенно свойство 2) $\lambda$-системы.

	$\Leftarrow$ Проверим сначала, что $\mathcal{F}$ -- алгебра. Свойства $1),\,3)$ уже есть. По свойству $2)$ $\lambda$-системы $\overline{A} = \Omega \setminus A \in \mathcal{F}$, если $A \in \mathcal{F}$. Значит $\mathcal{F}$ -- алгебра.

	Пусть $\{A_n,\, n \in \mathbb{N}\},\, A_n \in \mathcal{F}$. Рассмотрим $B_n : B_1 = A_1, B_i = A_i \setminus \left(\cup_{k = 1}^{i - 1} A_k\right)$. Тогда: $\forall n \: B_n \in \mathcal{F},\, \forall i \neq j \: B_i \cap B_j = \emptyset$. Рассмотрим $C_n = \sqcup_{m = 1}^n B_m \in \mathcal{F}$. Тогда $C_n \subset C_{n + 1}$ и $\cup_{n = 1}^\infty C_n = \sqcup_{n = 1}^\infty B_n \Rightarrow$ по $3)$ свойству $\lambda$-системы: $C_n \uparrow \sqcup_{n = 1}^\infty B_n \in \mathcal{F}$.
\end{proof}

\begin{lemma}
	Пусть $\mathcal{M}$ -- система подмножеств $\Omega$. Тогда существует минимальная (по включению) $\sigma$-алгебра (алгебра, $\pi$-система, $\lambda$-система), обозначаемая $\sigma(\mathcal{M})$ ($\alpha(\mathcal{M}),\, \pi(\mathcal{M}),\, \lambda(\mathcal{M})$), содержащая $\mathcal{M}$.
\end{lemma}

\begin{example}
	\begin{enumerate}
		\item Если $\Omega = \mathbb{R}$, то борелевской $\sigma$-алгеброй на $\mathbb{R}$ называется наименьшая $\sigma$-алгебра, содержащая все интервалы
		      \[\mathcal{B}(\mathbb{R}) = \sigma((a;\;b) ,\: a < b)\]
		\item Если $\Omega = \mathbb{R}^n,\, n > 1$.

		      Борелевской $\sigma$-алгеброй в $\mathbb{R}^n$ называется минимальная $\sigma$-алгебра, содержащая множества вида $B_1 \times \cdots \times B_n,\, B_i \in \mathcal{B}(\mathbb{R})$, то есть
		      \[\mathcal{B}(\mathbb{R}^n) = \sigma(B_1 \times \cdots \times B_n:\: B_i \in \mathcal{B}(\mathbb{R}))\]
		\item Если $\Omega = \mathbb{R}^\infty$, то есть $\Omega$ содержит все счётные последовательности вещественных чисел.

		      Для $n \in \mathbb{N}$ и $B_n \in \mathcal{B}(\mathbb{R}^n)$ введём циллиндр:
		      \[F_n(B_n) = \{\vec{x} \in \mathbb{R}^\infty :\: (x_1,\,\cdots,\,x_n) \in B_n\}\]
		      Тогда минимальная $\sigma$-алгебра, содержащая все циллиндры, называется борелевской в $\mathbb{R}^\infty$, то есть
		      \[\mathcal{B}(\mathbb{R}^\infty) = \sigma(F_n(B_n):\: n \in \mathbb{N},\, B_n \in \mathcal{B}(\mathbb{R}^n))\]
	\end{enumerate}
\end{example}
