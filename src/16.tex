\section{Достаточное условие сходимости с вероятностью\dots}
\begin{lemma}
	Достаточное условие сходимости с вероятностью 1.

	Если
	\[\forall \varepsilon > 0:\: \sum_{n = 1}^\infty P(|\xi_n - \xi| \geq \varepsilon) < +\infty\]
	то $\xi_n \stackrel{\text{п.н.}}{\to} \xi$
\end{lemma}

\begin{proof}
	Рассмотрим
	\begin{align*}
		P(\sup_{k \geq n}|\xi_k - \xi| > \varepsilon) = P\left(\bigcup_{k = n}^\infty \{|\xi_k - \xi| > \varepsilon\}\right) \leq \\
		\sum_{k = n}^\infty P(|\xi_k - \xi| > \varepsilon) \leq \sum_{k = n}^\infty P(|\xi_k - \xi| \geq \varepsilon) \stackrel{n \to +\infty}{\to} 0
	\end{align*}
	В силу стремления остатка сходящегося ряда к нулю.

	Тогда по критерию $\xi_n \stackrel{\text{п.н.}}{\to} \xi$
\end{proof}

\begin{corollary}
	Если $\xi_n \stackrel{P}{\to} \xi$, то $\exists$ подпоследовательность $\{\xi_{n_k},\, k \in \mathbb{N}\}$, такая что
	\[\xi_{n_k} \stackrel{\text{п.н},\, k \to +\infty}{\to} \xi\]
\end{corollary}

\begin{proof}
	Выберем $n_k$ так, чтобы $n_k > n_{k - 1}$ и
	\[P(|\xi_{n_k} - \xi| \geq \frac{1}{k}) \leq 2^{-k}\]
	выбор возможен в силу сходимости по вероятности.

	Проверим достаточное условие: пусть $\varepsilon > 0$, выберем $k_0 > \frac{1}{\varepsilon}$. Тогда
	\[
		\sum_{k = k_0}^\infty P(|\xi_{n_k} - \xi| \geq \varepsilon) \leq \sum_{k = k_0}^\infty P(|\xi_{n_k} - \xi| \geq \frac{1}{k}) \leq \sum_{k = k_0}^\infty 2^{-k} < +\infty \Rightarrow \xi_{n_k} \stackrel{\text{п.н.},\, k \to +\infty}{\to} \xi
	\]
\end{proof}

\begin{theorem}
	УЗБЧ в форме Кантелли.

	Пусть $\{\xi_n,\, n \in \mathbb{N}\}$ -- это независимые случайные величины, такие, что
	\[\exists c > 0 \: \forall n \in \mathbb{N}:\: E(\xi_n - E\xi_n)^4 \leq c\]
	Обозначим $S_n = \xi_1 + \cdots + \xi_n$. Тогда
	\[\frac{S_n - ES_n}{n} \stackrel{\text{п.н.},\, n \to +\infty}{\to} 0\]
\end{theorem}

\begin{proof}
	Без ограничения общности считаем, что $\forall n \in \mathbb{N}:\: E\xi_n = 0$, иначе рассмотрим
	\[\xi_n' = \xi_n - E\xi_n\]
	Хотим проверить достаточное условие. Для $\varepsilon > 0$:
	\[P\left(\left|\frac{S_n}{n}\right| \geq \varepsilon\right) = P\left(\frac{S_n^4}{n^4} \geq \varepsilon ^4\right) \stackrel{\text{н-во Маркова}}{\leq} \frac{ES_n^4}{\varepsilon^4n^4}\]
	Но
	\[ES_n^4 = \sum_{i,\,j,\,k,\,l = 1}^n E\xi_i\xi_j\xi_k\xi_l = \sum_{i = 1}^n ES_i^4 + 6\sum_{i < j}ES_i^2ES_j^2\]
	По условию $\forall i \in \mathbb{N}:\: ES_i^4 \leq c \Rightarrow \forall i \in \mathbb{N}:\: E\xi_i^2 \leq \sqrt{E\xi_i^4} \leq \sqrt{c} \Rightarrow$
	\[ES_n^4 \leq n\cdot c + 6\cdot c\cdot C_n^2 = O(n^2) \Rightarrow \frac{ES_n^4}{\varepsilon^4n^4} = O\left(\frac{1}{n^2}\right)\]
	Значит ряд сходится и работает достаточно условие сходимости с вероятностью 1.
\end{proof}

\begin{note}
	Смысл УЗБЧ.

	Теоретическое обоснование принципа устойчивых частот. Пусть
	\[\xi_i = \mathbb{I}\{A \text{ произошло в }i \text{-ом эксперименте}\}\]
	Тогда частота появления $A$ стремится к:
	\[\nu_n(A) = \frac{\xi_1 + \cdots + \xi_n}{n} \stackrel{\text{п.н.}}{\to} E\xi_1 = P(A)\]
\end{note}
